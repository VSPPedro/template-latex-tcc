% ---
% Pacotes básicos 
% ---
\usepackage{lmodern}			% Usa a fonte Latin Modern			
\usepackage[T1]{fontenc}		% Selecao de codigos de fonte.
\usepackage[utf8]{inputenc}		% Codificacao do documento (conversão automática dos acentos)
\usepackage{lastpage}			% Usado pela Ficha catalográfica
\usepackage{indentfirst}		% Indenta o primeiro parágrafo de cada seção.
\usepackage{color}				% Controle das cores
\usepackage{graphicx}			% Inclusão de gráficos
\usepackage{microtype} 			% para melhorias de justificação
\usepackage{fancyvrb}
\usepackage{booktabs}           % Para tabelas
\usepackage{adjustbox}          % Para ajustar tabelas
% ---
		
% ---
% Pacotes adicionais, usados apenas no âmbito do Modelo Canônico do abnteX2
% ---
\usepackage{lipsum}				% para geração de dummy text
% ---

% ---
% Pacotes de citações
% ---
\usepackage[brazilian,hyperpageref]{backref}	 % Paginas com as citações na bibl
\usepackage[alf,abnt-etal-cite=1]{abntex2cite}
% --\usepackage[alf]{abntex2cite}	% Citações padrão ABNT
% -- \citeoption{abnt-etal-cite=2}

% --- 
% CONFIGURAÇÕES DE PACOTES
% --- 

% ---
% Configurações do pacote backref
% Usado sem a opção hyperpageref de backref
\renewcommand{\backrefpagesname}{Citado na(s) página(s):~}
% Texto padrão antes do número das páginas
\renewcommand{\backref}{}
% Define os textos da citação
\renewcommand*{\backrefalt}[4]{
	\ifcase #1 %
		Nenhuma citação no texto.%
	\or
		Citado na página #2.%
	\else
		Citado #1 vezes nas páginas #2.%
	\fi}%
% ---

% ---
% Informações de dados para CAPA e FOLHA DE ROSTO
% ---
\titulo{<Título do trabalho>}
\autor{<Nome do autor>}
\local{<Cidade(Ex: João Pessoa)>}
\data{<Mês e ano do trabalho(Ex: Junho de 2018)>}
\orientador{<Nome do(a) orientador(a)>}

% - Supervisor
\newcommand{\supervisorname}{Supervisor: }

\providecommand{\imprimirsupervisorRotulo}{}
\providecommand{\imprimirsupervisor}{}

\newcommand{\supervisor}[2][\supervisorname]%
  {\renewcommand{\imprimirsupervisorRotulo}{#1}%
   \renewcommand{\imprimirsupervisor}{#2}}

\supervisor{<Nome do(a) supervisor(a)>}

% - Coordenador
\newcommand{\coordenadorname}{Coordenador: }

\providecommand{\imprimircoordenadorRotulo}{}
\providecommand{\imprimircoordenador}{}

\newcommand{\coordenador}[2][\coordenadorname]%
  {\renewcommand{\imprimircoordenadorRotulo}{#1}%
   \renewcommand{\imprimircoordenador}{#2}}
   
\coordenador{<Nome do(a) coordenador(a)>}

\instituicao{%
  INSTITUTO FEDERAL DA PARAÍBA
  \par
  UNIDADE ACADÊMICA DE INFORMAÇÃO E COMUNICAÇÃO
  \par
  <COORDENAÇÂO DO CURSO>}
\tipotrabalho{Relatório de Estágio}
% O preambulo deve conter o tipo do trabalho, o objetivo, 
% o nome da instituição e a área de concentração 
\preambulo{Relatório de Estágio Supervisionado apresentado à Coordenação do Curso Superior de Tecnologia em Sistemas para Internet do Instituto Federal de Educação, Ciência e Tecnologia da Paraíba como requisito parcial para obtenção do grau de Tecnólogo em Tecnologia em Sistemas para Internet.}
% ---


% ---
% Configurações de aparência do PDF final

% alterando o aspecto da cor azul
\definecolor{blue}{RGB}{0,0,0}

% informações do PDF
\makeatletter
\hypersetup{
     	%pagebackref=true,
		pdftitle={\@title}, 
		pdfauthor={\@author},
    	pdfsubject={\imprimirpreambulo},
	    pdfcreator={LaTeX with abnTeX2},
		pdfkeywords={abnt}{latex}{abntex}{abntex2}{trabalho acadêmico}, 
		colorlinks=true,       		% false: boxed links; true: colored links
    	linkcolor=blue,          	% color of internal links
    	citecolor=blue,        		% color of links to bibliography
    	filecolor=magenta,      		% color of file links
		urlcolor=blue,
		bookmarksdepth=4
}
\makeatother
% --- 

% --- 
% Espaçamentos entre linhas e parágrafos 
% --- 

% O tamanho do parágrafo é dado por:
\setlength{\parindent}{1.3cm}

% Controle do espaçamento entre um parágrafo e outro:
\setlength{\parskip}{0.2cm}  % tente também \onelineskip


% Novo list of (listings) para QUADROS
\newcommand{\quadroname}{Quadro}
\newcommand{\listofquadrosname}{Lista de quadros}

\newfloat[chapter]{quadro}{loq}{\quadroname}
\newlistof{listofquadros}{loq}{\listofquadrosname}
\newlistentry{quadro}{loq}{0}

% configurações para atender às regras da ABNT
\counterwithout{quadro}{chapter}
\renewcommand{\cftquadroname}{\quadroname\space} 
\renewcommand*{\cftquadroaftersnum}{\hfill--\hfill}

% Configuração de posicionamento padrão:
\setfloatlocations{quadro}{hbtp}


%--- Configura a fonte como Time News Roman
%--- \usepackage{blindtext}
%--- \usepackage{mathptmx}
%-- \renewcommand{\sfdefault}{\rmdefault} % No serif fonts, use roman (times)